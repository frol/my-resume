%% start of file `template.tex'.
%% Copyright 2006-2011 Xavier Danaux (xdanaux@gmail.com).
%
% This work may be distributed and/or modified under the
% conditions of the LaTeX Project Public License version 1.3c,
% available at http://www.latex-project.org/lppl/.

\documentclass[11pt,a4paper,roman]{moderncv}   % possible options include font size ('10pt', '11pt' and '12pt'), paper size ('a4paper', 'letterpaper', 'a5paper', 'legalpaper', 'executivepaper' and 'landscape') and font family ('sans' and 'roman')

% moderncv themes
\moderncvstyle{classic}                        % style options are 'casual' (default) and 'classic'
\moderncvcolor{blue}                          % color options 'blue' (default), 'orange', 'green', 'red', 'purple', 'grey' and 'black'
%\renewcommand{\familydefault}{\sfdefault}    % to set the default font; use '\sfdefault' for the default sans serif font, '\rmdefault' for the default roman one, or any tex font name
%\nopagenumbers{}                             % uncomment to suppress automatic page numbering for CVs longer than one page

% NOTE: Don't forget to remove 'sans' from the \documentclass when using fontspec
%\usepackage{fontspec}

%\defaultfontfeatures{Ligatures=TeX}

%\newfontfamily{\booktitle}[
%    Path=fonts/,
%    Ligatures=TeX,
%]{Neocyr.otf}

%\newfontfamily{\mainfonttt}[
%\setmainfont[
%    Path=fonts/,
%    UprightFont=*-Regular.ttf,
%    BoldFont=*-Bold.ttf,
%    ItalicFont=*-Italic.ttf,
%    Ligatures=TeX
%]{Academy}

%\setmainfont{Liberation Serif}
%\setmainfont{Purisa}
%\setmainfont{TeX Gyre Chorus}
%\setmainfont{Rock Salt}
%\setmainfont{Neocyr}

% adjust the page margins
\usepackage[scale=0.84]{geometry}
\setlength{\hintscolumnwidth}{3cm}           % if you want to change the width of the column with the dates

% personal data
\firstname{\textsc{Vladyslav}}
\familyname{\textsc{Frolov}}
\title{Software Engineer}               % optional, remove the line if not wanted
\address{World citizen, originally from Ukraine}    % optional, remove the line if not wanted
%\phone{+38~(067)~571~63~98}                     % optional, remove the line if not wanted
%\mobile{+2~(345)~678~901}                      % optional, remove the line if not wanted
%\fax{+3~(456)~789~012}                        % optional, remove the line if not wanted
\email{frolvlad@gmail.com}                          % optional, remove the line if not wanted
%\homepage{prostoksi.com}                    % optional, remove the line if not wanted
\extrainfo{%
    \faGithub{} \httplink{github.com/frol}%
}            % optional, remove the line if not wanted
%\photo[64pt][0.4pt]{graphics/photo}                  % '64pt' is the height the picture must be resized to, 0.4pt is the thickness of the frame around it (put it to 0pt for no frame) and 'picture' is the name of the picture file; optional, remove the line if not wanted
%\quote{Some quote (optional)}                 % optional, remove the line if not wanted

\moderncvicons{awesome}
\ifxetexorluatex
    \renewcommand*{\labelitemi}{\strut\textcolor{color1}{\raisebox{1pt}{\tiny\faCircleO}}}
\fi

% This is a fake small-caps implementation (\textsc) to support bold small-caps
\newcommand\fakesc[1]{\fakeschelper#1 \relax\relax}
\def\fakeschelper#1 #2\relax{%
  \fakeschelphelp#1\relax\relax%
  \if\relax#2\relax\else\ \fakeschelper#2\relax\fi%
}
\def\Hscale{.83}\def\Vscale{.72}\def\Cscale{1.00}
\def\fakeschelphelp#1#2\relax{%
  \ifnum`#1>``\ifnum`#1<`\{\scalebox{\Hscale}[\Vscale]{\uppercase{#1}}\else%
    \scalebox{\Cscale}[1]{#1}\fi\else\scalebox{\Cscale}[1]{#1}\fi%
  \ifx\relax#2\relax\else\fakeschelphelp#2\relax\fi}

\newcommand\textscbf[1]{\textbf{\textrm{\fakesc{#1}}}}

% to show numerical labels in the bibliography (default is to show no labels); only useful if you make citations in your resume
%\makeatletter
%\renewcommand*{\bibliographyitemlabel}{\@biblabel{\arabic{enumiv}}}
%\makeatother

% bibliography with mutiple entries
%\usepackage{multibib}
%\newcites{book,misc}{{Books},{Others}}
%----------------------------------------------------------------------------------
%            content
%----------------------------------------------------------------------------------
\begin{document}
\maketitle
\vspace*{-3em}
%\setlength{\parskip}{0.7em}

\section{\textsc{Background}}

Throughout my career, I have been into distributed systems, machine learning,
web, desktop, and kernel developments, and I am eager to learn more and I love
helping others to learn more.

\section{\textsc{Professional Skills}}
\cvitem{Programming Languages}{\small\textsc{%
    Rust $\circ$ Python $\circ$ Cython $\circ$
    TypeScript $\circ$ JavaScript (ES6+) $\circ$ Bash $\circ$ C++ $\circ$ C \ldots
}}
\cvitem{Frameworks and Tools}{\small\textsc{%
    Flask $\circ$ Django $\circ$ ReactJS $\circ$
    Dask $\circ$ Celery $\circ$ PySpark $\circ$ Apache Spark $\circ$
    NGINX $\circ$ Docker $\circ$ VIM $\circ$ Git $\circ$ OpenVPN \ldots
}}
\cvitem{Data Storages}{\small\textsc{%
    PostgreSQL $\circ$ MySQL $\circ$ SQLite $\circ$
    RabbitMQ $\circ$ Redis $\circ$ MongoDB $\circ$ Memcache
}}
\cvitem{OS}{\small\textsc{%
    BTW, I use Arch Linux
}}


\section{\textsc{Education}}
\cventry{2008 -- 2013}{Bachelor's \& Master's Degree}{Kharkiv National University of Radio Electronics}{}{}{%
    \textit{First Class Honours} \\
    Security of Information and Communication Systems \\
    \textsc{``Research of Synthesis Methods of Nonlinear S-boxes for Block Symmetric Ciphers''}}  % arguments 3 to 6 can be left empty

\cventry{2005 -- 2009}{Self-education}{}{}{}{%
    I was taught algorithms and data structures in Pascal, and so I competed in
    a number of individual and team competitive programming contests. Later, I
    learned about desktop development and network design, which allowed me to
    implement a Direct Connect (DC) server from scratch and host a DC hub on a
    local city network. \\ \vspace{-0.6em}\\
    \textsc{\color{darkgray} -- Algorithms \& Data Structures, Pascal, Delphi, C++, Qt, Linux}
}

\section{\textsc{Experience}}

%\subsection{Company}
%\cventry{year--year}{Job title}{Employer}{City}{}{Description line 1\newline{}Description line 2}

\subsection{NEAR inc. (near.org)}
\cventry{2019 -- today}{Software Engineer Independent Contractor}{Startup company}{remote}{}{%
    Develop a sharded, developer-friendly, proof-of-stake public blockchain (in Rust) and developer
    tools (in TypeScript). Given the startup nature, I am involved in all kinds of work on the way
    to the product delivery, though, mostly stick to technical problems.
}

\subsection{Mirabit}
\cventry{2017 -- 2018}{Independent Consultant}{Software Engineering company}{Kharkiv, Ukraine}{}{%
    Consulted on the BigData in Machine Learning projects.
}

\vspace{-0em}
\subsection{Salford Systems}
\cventry{2012 -- 2017}{Software Engineer Independent Contractor}{Data Mining and Predictive Analytics Software company}{remote}{}{%
    Designed, led, and developed distributed data mining projects which handle big data loads.
    The team I led was in charge of solving all the technical problems (development and operations). \\ \vspace{-1em}
    \begin{itemize}
        \item Backend: \textsc{\color{darkgray}
                \mbox{Python}, \mbox{Cython}, \mbox{Python C-extensions}, \mbox{C},
                \mbox{Flask}, \mbox{Django},
                \mbox{Dask}, \mbox{PySpark}, \mbox{Celery},
                \mbox{Hadoop HDFS},
                \mbox{PostgreSQL}, \mbox{RabbitMQ}, \mbox{Redis}
            } \\ \vspace{-0.9em}
        \item Frontend: \textsc{\color{darkgray}
                \mbox{JavaScript (ES6)}, \mbox{ReactJS}, \mbox{Redux}
            } \\ \vspace{-0.9em}
        \item Operations: \textsc{\color{darkgray}
                \mbox{AWS}, \mbox{Docker}, \mbox{Hadoop},
                \mbox{NGINX} \ldots
            }
    \end{itemize}
}

\vspace{0.15em}
\subsection{Open-source (github.com/frol)}
\cventry{2009 -- today}{Contributor}{open-source software}{remote}{}{%
    I became a fan of open-source software the moment I consciously used it for the first time back in 2009, developing a small project in Qt.
    At that time, I had already been using Linux for over two years, but most of my projects were written from scratch, meaning that I didn't use any thirdparty libraries. \\ \vspace{-0.7em}\\
    It was September 29, 2009, when the first project I published to the open-source was born.
    Since then I have open-sourced over 30 projects and contributed to over 100, including Linux kernel, Docker, and Django. \\ \vspace{-0.7em}\\
    I am an active maintainer of some packages on \href{https://hub.docker.com/u/frolvlad}{\underline{Docker Hub}}
    and \href{https://github.com/frol?tab=repositories&q=feedstock}{\underline{conda-forge} (Anaconda Python)}. \\ \vspace{-0.7em}\\
    My recent contributions are around Rust packages (cgroups integration, flatbuffers helpers, API ergonomics). \\ \vspace{-0.7em}\\
    \textsc{\color{darkgray} -- Git, Rust, Python, JS, Linux, Docker, Heroku, Travis CI, CircleCI}
}

\vspace{0.35em}
\subsection{Escalibro (escalibro.com)}
\cventry{2010 -- today}{Co-creator}{pet project}{}{}{%
    This is a Web-platform for writers and readers, that provides simple and convinient book writing and reading.
    I was involved in the every bit of the project: servers tuning, UI design, frontend and backend development. \\ \vspace{-0.6em}\\
    \textsc{\color{darkgray} -- Python, Django, Celery, NGINX, MySQL, RabbitMQ, HTML/CSS/JS, Linux}
}

\vspace{0.35em}
\subsection{Other Software Engineering Companies}
\cventry{2009 -- 2012}{Software Engineer Independent Contractor}{}{Kharkiv, Ukraine}{}{%
    Developed a variety of websites and web-services \\ \vspace{-0.6em}\\
    \textsc{\color{darkgray} -- Python, Django, MySQL, HTML/CSS/JS}
}

\section{\textsc{Other Activities}}

\vspace{0.1em}
\subsection{PeerLab Rust}
\cventry{2018 -- 2019 (biweekly)}{Co-organizer \& Speaker}{}{Kyiv \& Kharkiv, Ukraine}{}{%
    Meet-ups and Tech-Talks for Rust/C/C++ developers in Ukraine.}

\vspace{0.1em}
\subsection{Q-BIT}
\cventry{2015 -- today}{Speaker \& Mentor}{}{Kharkiv, Ukraine}{}{%
    Q-BIT is a local non-profit youth science community, where I give lectures and mentor school
    students learning software engineering, including algorithms and data structures.}

\vspace{0.1em}
\subsection{Seventh All-Ukrainian Blockchain Hackathon}
\cventry{September\,2019}{Mentor}{}{Kyiv, Ukraine}{}{}


\section{\textsc{Languages}}
\cvitem{Native}{Russian, Ukrainian}
\cvitem{Fluent}{English}
\cvitem{Bootstrapping}{Spanish}

%\section{Interests}
%\cvitem{hobby 1}{Description}
%\cvitem{hobby 2}{Description}
%\cvitem{hobby 3}{Description}

%\section{Extra 1}
%\cvlistitem{Item 1}
%\cvlistitem{Item 2}
%\cvlistitem{Item 3}

%\renewcommand{\listitemsymbol}{-~}            % change the symbol for lists

%\section{Extra 2}
%\cvlistdoubleitem{Item 1}{Item 4}
%\cvlistdoubleitem{Item 2}{Item 5\cite{book1}}
%\cvlistdoubleitem{Item 3}{}

% Publications from a BibTeX file without multibib\renewcommand*{\bibliographyitemlabel}{\@biblabel{\arabic{enumiv}}}% for BibTeX numerical labels
%\nocite{*}
%\bibliographystyle{plain}
%\bibliography{publications}                   % 'publications' is the name of a BibTeX file

% Publications from a BibTeX file using the multibib package
%\section{Publications}
%\nocitebook{book1,book2}
%\bibliographystylebook{plain}
%\bibliographybook{publications}              % 'publications' is the name of a BibTeX file
%\nocitemisc{misc1,misc2,misc3}
%\bibliographystylemisc{plain}
%\bibliographymisc{publications}              % 'publications' is the name of a BibTeX file

%\clearpage\end{CJK*}                         % if you are typesetting your resume in Chinese using CJK; the \clearpage is required for fancyhdr to work correctly with CJK, though it kills the page numbering by making \lastpage undefined
\end{document}


%% end of file `template.tex'.
