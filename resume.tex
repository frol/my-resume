%% start of file `template.tex'.
%% Copyright 2006-2011 Xavier Danaux (xdanaux@gmail.com).
%
% This work may be distributed and/or modified under the
% conditions of the LaTeX Project Public License version 1.3c,
% available at http://www.latex-project.org/lppl/.

\documentclass[11pt,a4paper,sans]{moderncv}   % possible options include font size ('10pt', '11pt' and '12pt'), paper size ('a4paper', 'letterpaper', 'a5paper', 'legalpaper', 'executivepaper' and 'landscape') and font family ('sans' and 'roman')

% moderncv themes
\moderncvstyle{classic}                        % style options are 'casual' (default) and 'classic' 
\moderncvcolor{blue}                          % color options 'blue' (default), 'orange', 'green', 'red', 'purple', 'grey' and 'black'
%\renewcommand{\familydefault}{\sfdefault}    % to set the default font; use '\sfdefault' for the default sans serif font, '\rmdefault' for the default roman one, or any tex font name
%\nopagenumbers{}                             % uncomment to suppress automatic page numbering for CVs longer than one page

% NOTE: Don't forget to remove 'sans' from the \documentclass when using fontspec
%\usepackage{fontspec}

%\defaultfontfeatures{Ligatures=TeX}

%\newfontfamily{\booktitle}[
%    Path=fonts/,
%    Ligatures=TeX,
%]{Neocyr.otf}

%\newfontfamily{\mainfonttt}[
%\setmainfont[
%    Path=fonts/,
%    UprightFont=*-Regular.ttf,
%    BoldFont=*-Bold.ttf,
%    ItalicFont=*-Italic.ttf,
%    Ligatures=TeX
%]{Academy}

%\setmainfont{Liberation Serif}
%\setmainfont{Purisa}
%\setmainfont{TeX Gyre Chorus}
%\setmainfont{Rock Salt}
%\setmainfont{Neocyr}

% adjust the page margins
\usepackage[scale=0.84]{geometry}
\setlength{\hintscolumnwidth}{3cm}           % if you want to change the width of the column with the dates

% personal data
\firstname{\textsc{Vladyslav}}
\familyname{\textsc{Frolov}}
\title{Software Engineer}               % optional, remove the line if not wanted
\address{Kyiv, Ukraine}    % optional, remove the line if not wanted
%\phone{+38~(067)~571~63~98}                     % optional, remove the line if not wanted
%\mobile{+2~(345)~678~901}                      % optional, remove the line if not wanted
%\fax{+3~(456)~789~012}                        % optional, remove the line if not wanted
\email{frolvlad@gmail.com}                          % optional, remove the line if not wanted
%\homepage{prostoksi.com}                    % optional, remove the line if not wanted
\extrainfo{%
    \faGithub{} \httplink{github.com/frol}%
}            % optional, remove the line if not wanted
%\photo[64pt][0.4pt]{graphics/photo}                  % '64pt' is the height the picture must be resized to, 0.4pt is the thickness of the frame around it (put it to 0pt for no frame) and 'picture' is the name of the picture file; optional, remove the line if not wanted
%\quote{Some quote (optional)}                 % optional, remove the line if not wanted

\moderncvicons{awesome}
\ifxetexorluatex
    \renewcommand*{\labelitemi}{\strut\textcolor{color1}{\raisebox{1pt}{\tiny\faCircleO}}}
\fi

% This is a fake small-caps implementation (\textsc) to support bold small-caps
\newcommand\fakesc[1]{\fakeschelper#1 \relax\relax}
\def\fakeschelper#1 #2\relax{%
  \fakeschelphelp#1\relax\relax%
  \if\relax#2\relax\else\ \fakeschelper#2\relax\fi%
}
\def\Hscale{.83}\def\Vscale{.72}\def\Cscale{1.00}
\def\fakeschelphelp#1#2\relax{%
  \ifnum`#1>``\ifnum`#1<`\{\scalebox{\Hscale}[\Vscale]{\uppercase{#1}}\else%
    \scalebox{\Cscale}[1]{#1}\fi\else\scalebox{\Cscale}[1]{#1}\fi%
  \ifx\relax#2\relax\else\fakeschelphelp#2\relax\fi}

\newcommand\textscbf[1]{\textbf{\textrm{\fakesc{#1}}}}

% to show numerical labels in the bibliography (default is to show no labels); only useful if you make citations in your resume
%\makeatletter
%\renewcommand*{\bibliographyitemlabel}{\@biblabel{\arabic{enumiv}}}
%\makeatother

% bibliography with mutiple entries
%\usepackage{multibib}
%\newcites{book,misc}{{Books},{Others}}
%----------------------------------------------------------------------------------
%            content
%----------------------------------------------------------------------------------
\begin{document}
\maketitle
%\vspace*{-2em}
%\setlength{\parskip}{0.7em}

\section{\textsc{Background}}

My journey to computer science started at school where I was introduced to
competitive programming. Thus, I learned the foundation of algorithms and data
structures, and competed in a number of programming contests using Pascal. The
algorithmic foundation helped me to solve complex problems and solve them
optimaly. Later, I tought myself C, C++, Python, JS, Rust, and other programming
languages and technologies. I was into desktop, web, kernel, distributed
systems, and machine learning developments, and I want to learn more and I love
helping others to learn more.

\vspace*{1em}
\section{\textsc{Education}}
\cventry{September\,2012 -- June\,2013}{Master's Degree}{Kharkiv National University of Radio Electronics}{}{\textit{First Class Honours}}{Security of Information and Communication Systems \\ \textsc{``Research of Synthesis Methods of Nonlinear S-boxes for Block Symmetric Ciphers''}}  % arguments 3 to 6 can be left empty
\cventry{September\,2008 -- June\,2012}{Bachelor's Degree}{Kharkiv National University of Radio Electronics}{}{\textit{First Class Honours}}{Security of Information and Communication Systems \\ \textsc{``Security Analysis of WEP Protocol in 802.11 WLAN Computer Networks''}}

\vspace*{1em}
\section{\textsc{Experience}}

%\subsection{Company}
%\cventry{year--year}{Job title}{Employer}{City}{}{Description line 1\newline{}Description line 2}

\vspace{0.35em}
\subsection{Open-source (github.com/frol)}
\cventry{September\,2009 -- today}{Contributor}{open-source software}{}{}{%
    I became a fan of open-source software the moment I consciously used it for the first time back in 2009, developing a small project in Qt.
    At that time, I had already been using Linux for over two years, but most of my projects were written from scratch, meaning that I didn't use any thirdparty libraries. \vspace{0.4em}\\ %
    It was September 29, 2009, when the first project I published to the open-source was born.
    Since then I have open-sourced over 30 projects and contributed to over 100, including Linux kernel, Android for x86, Docker, and Django.
}

\vspace{0.35em}
\subsection{Q-BIT (qbit.org.ua)}
\cventry{March\,2015 -- today}{Education Evangelist}{local science community}{}{}{%
    I used to be a student at Q-BIT, and now I am helping new students (giving
    lectures and guiding). Q-BIT has its own web-platform for training programming
    skills and conducting programming contests, and I have designed and developed
    the backend for it that does testing (compilation and execution) student
    solutions in over 15 different programming languages. \\ %
    \textsc{\color{darkgray} -- Linux, Docker, NGINX, Python, Rust, C, Bash, JS, C++, Pascal}
}

\vspace{0.35em}
\subsection{Escalibro (escalibro.com)}
\cventry{October\,2010 -- today}{Co-creator}{pet project}{}{}{%
    This is a Web-service for writers and readers, that provides simple and convinient book writing and reading.
    I was involved in the every bit of the project: servers tuning, UI design, frontend and backend development. \\ %
    \textsc{\color{darkgray} -- Linux, Python, Django, Celery, NGINX, MySQL, RabbitMQ, HTML/CSS/JS}
}

\vspace{0.35em}
\subsection{Salford Systems}
\cventry{June\,2012 -- March\,2018}{Software Engineer}{Data Mining and Predictive Analytics Software company}{}{}{%
    \begin{itemize}%
        \item Designed and developed distributed data mining (BigData) projects \\
            \textsc{\color{darkgray} --
                \mbox{Python}, \mbox{Cython}, \mbox{Dask}, \mbox{Yarn} \mbox{Hadoop},
                \mbox{PySpark}, \mbox{Python C-extensions}, \mbox{PyInvoke},
                \mbox{Anaconda}
            }
        \item Designed and developed data mining SaaS with a distributed backend in the cloud \\
            -- API Server: \textsc{\color{darkgray} \mbox{Python}, \mbox{Flask}, \mbox{PostgreSQL}, \mbox{SeaweedFS}} \\
            -- Frontend: \textsc{\color{darkgray} \mbox{JavaScript (ES6)}, \mbox{ReactJS}, \mbox{Redux}} \\
            -- Backend: \textsc{\color{darkgray}\mbox{Cython}, \mbox{Dask}, \mbox{RabbitMQ}, \mbox{Hadoop HDFS}} \\
            -- Other: \textsc{\color{darkgray} \mbox{Docker}, \mbox{Django}, \mbox{Redis}, \mbox{AWS}}
        \item Developed an automated distributed testing system for Salford Systems' software suite \\
            \textsc{\color{darkgray} -- \mbox{Python}, \mbox{Django}, \mbox{Celery}, \mbox{PostgreSQL}, \mbox{Redis}}
        \item Developed a text mining tool using Natural Language Toolkit \\
            \textsc{\color{darkgray} -- \mbox{Python}, \mbox{NLTK}}
        \item Set up Salford Systems IT infrastructures \\
            \textsc{\color{darkgray} --
                \mbox{Linux}, \mbox{Docker}, \mbox{Hadoop}, \mbox{FreeIPA},
                \mbox{Rancher}, \mbox{GitLab}, \mbox{Mattermost}, \mbox{CFEngine},
                \mbox{NGINX}, \mbox{Jenkins}, \mbox{Redmine}, \mbox{PyInvoke}
            }
    \end{itemize}
}

\vspace{0.35em}
\subsection{Fahrenheit 451}
\cventry{April\,2011 -- May\,2012}{Software Engineer}{Website building company}{}{}{%
    Developed a food delivery web-service for restaurant chains ``Mafia''
    (\url{mafia.ua}) and ``Yakitoria'' (\url{yaki.kh.ua}) \\ %
    \textsc{\color{darkgray} -- Python, Django, MySQL, HTML/CSS/JS}
}

\vspace{0.35em}
\subsection{EwaDev}
\cventry{October\,2009 -- December\,2011}{Software Engineer}{Website building company}{}{}{%
    Developed a variety of websites and web-services \\ %
    \textsc{\color{darkgray} -- Linux, Python, Django, MySQL, HTML/CSS/JS}
}

\section{\textsc{Computer skills} \small{\textit{(my preferences are marked in bold)}}}
\cvitem{Programming Languages}{\small\textsc{%
    \textscbf{Rust} $\circ$ \textscbf{Python} $\circ$ \textscbf{Cython} $\circ$
    \textscbf{JavaScript (ES6+)} $\circ$ Bash $\circ$ C++ $\circ$ C $\circ$
    HTML $\circ$ SQL $\circ$ {\LaTeX} $\circ$ Delphi/Pascal
}}
\cvitem{Frameworks and Tools}{\small\textsc{%
    \textscbf{Flask} $\circ$ \textscbf{ReactJS} $\circ$ \textscbf{VIM} $\circ$
    \textscbf{Dask} $\circ$ Django $\circ$ Celery $\circ$ PySpark $\circ$
    Apache Spark $\circ$ Hadoop YARN $\circ$ jQuery $\circ$
    \textscbf{NGINX} $\circ$ Apache $\circ$ Anaconda $\circ$ Qt $\circ$
    \textscbf{Docker} $\circ$ \textscbf{Git} $\circ$ Jenkins $\circ$
    \textscbf{OpenVPN} $\circ$ StrongSWAN $\circ$ \ldots
}}
\cvitem{Databases}{\small\textsc{%
    \textscbf{PostgreSQL} $\circ$ MySQL $\circ$ \textscbf{SQLite} $\circ$
    \textscbf{RabbitMQ} $\circ$ \textscbf{Redis} $\circ$ MongoDB $\circ$ Memcache
}}
\cvitem{OS}{\small\textsc{%
    \textscbf{GNU/Linux} (Debian, Ubuntu, CentOS, \textscbf{ArchLinux}) $\circ$ MS Windows $\circ$ macOS}
}

\section{\textsc{Languages}}
\cvitem{Russian, Ukrainian}{Native}
\cvitem{English}{Fluent}

%\section{Interests}
%\cvitem{hobby 1}{Description}
%\cvitem{hobby 2}{Description}
%\cvitem{hobby 3}{Description}

%\section{Extra 1}
%\cvlistitem{Item 1}
%\cvlistitem{Item 2}
%\cvlistitem{Item 3}

%\renewcommand{\listitemsymbol}{-~}            % change the symbol for lists

%\section{Extra 2}
%\cvlistdoubleitem{Item 1}{Item 4}
%\cvlistdoubleitem{Item 2}{Item 5\cite{book1}}
%\cvlistdoubleitem{Item 3}{}

% Publications from a BibTeX file without multibib\renewcommand*{\bibliographyitemlabel}{\@biblabel{\arabic{enumiv}}}% for BibTeX numerical labels
%\nocite{*}
%\bibliographystyle{plain}
%\bibliography{publications}                   % 'publications' is the name of a BibTeX file

% Publications from a BibTeX file using the multibib package
%\section{Publications}
%\nocitebook{book1,book2}
%\bibliographystylebook{plain}
%\bibliographybook{publications}              % 'publications' is the name of a BibTeX file
%\nocitemisc{misc1,misc2,misc3}
%\bibliographystylemisc{plain}
%\bibliographymisc{publications}              % 'publications' is the name of a BibTeX file

%\clearpage\end{CJK*}                         % if you are typesetting your resume in Chinese using CJK; the \clearpage is required for fancyhdr to work correctly with CJK, though it kills the page numbering by making \lastpage undefined
\end{document}


%% end of file `template.tex'.
